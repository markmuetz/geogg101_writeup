\documentclass{article}

\usepackage{graphicx}
% Reduces margins substantially.
\usepackage{fullpage}

\usepackage[backend=bibtex,style=authoryear,sorting=nyt,dashed=false]{biblatex}
% Changes e.g. \parencite{jones2003} from (Jones 2003) to (Jones, 2003)
\renewcommand*{\nameyeardelim}{\addcomma\space}
\defbibfilter{practical1}{
    keyword=practical1
}

\defbibfilter{practical2}{
    keyword=practical2
}

\bibliography{references/references}

\begin{document}

% Practical 1:
\part*{Modelling the effects of dredging on a tidal estuary}

\section{Introduction}
According to \textcite{cea2012bathymetric} the effects of bathymatric uncertainty are well understood. However, it has been shown that a higher albedo leads to cooling.

\begin{figure}[hbp]
    \centering
    \includegraphics[width=\linewidth]{geogg101_screenshots/practical1/foweyROI}
    \caption{Composite satellite image of the Fowey Estuary}
    \label{fig:fowey_estuary}
\end{figure}

\section{Methods}

\section{Results}
Results to go here. As can be seen in Figure~\ref{fig:rising_tide}.

\begin{figure}[hbp]
    \centering
    \includegraphics[width=\linewidth]{geogg101_screenshots/practical1/global_rising_2_400}
    \caption{Rising tide in the Fowey Estuary}
    \label{fig:rising_tide}
\end{figure}

\section{Conclusions}
Conclusions to go here\ldots

\printbibliography[filter=practical1]

\newpage
\setcounter{section}{0}

% Practical 2:
\part*{Modelling the effects of deforestation on a catchment area}

\section{Introduction}
MIKE SHE is a deterministic, distributed and physically based modelling system for modelling water flow through catchments \parencite{refshaard1995mike}. It can be coupled to MIKE 11, which models one-dimensional water flow through channels \parencite{havno1995mike}. Wetlands can be effectively simulated by MIKE SHE \parencite{thompson2004simulation}. 

\section{Methods}
Methods

\section{Results}
Results to go here\ldots

\section{Conclusions}
Conclusions to go here\ldots

\printbibliography[filter=practical2]


\end{document}
