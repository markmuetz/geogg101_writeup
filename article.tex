\documentclass{article}

% Some useful packages.
\usepackage{amsmath}
\usepackage{siunitx}
\usepackage{graphicx}
\usepackage{verbatim}

% Reduces margins substantially.
\usepackage{geometry}
\newgeometry{margin=2.5cm}

% Allows headers and footers.
\usepackage{fancyhdr}
\pagestyle{fancy}
% Get rid of annoying line under header.
\renewcommand{\headrulewidth}{0pt}

\lhead{BTTM8 - SURFACE WATER MODELLING - EFFECTS OF DREDGING}
\chead{}
\rhead{}
%\lfoot{}
%\cfoot{}
%\rfoot{}

% Configure References sections. 2 filters are for the 2 sections. (1/practical)
\usepackage[backend=bibtex,style=authoryear,sorting=nyt,dashed=false]{biblatex}
% Changes e.g. \parencite{jones2003} from (Jones 2003) to (Jones, 2003)
\renewcommand*{\nameyeardelim}{\addcomma\space}
\defbibfilter{practical1}{keyword=practical1}
\defbibfilter{practical2}{keyword=practical2}
\bibliography{references/references}

\begin{document}

% Practical 1:
\part*{Modelling the effects of dredging on a tidal estuary}

\section{Introduction}

The purpose of this study is to implement a surface water model of the Fowey Estuary. This model is then used to characterise the existing hydrodynamic regime of the Estuary, analyse the sensitivity of the model to changes in the bed friction and evaluate the changes to the regime under a dredging scenario. % drop sensitivity?

The Fowey Estuary is a macrotidal estuary with a mean spring tidal range of \SI{4.8}{\m} and a tidal length of \SI{12}{km} \parencite{uncles2002dependence}. It is located in Cornwall on the south coast of England (see Figure \ref{fig:fowey_estuary_map}). It is fed by two rivers, the Fowey and the Lerryn, with mean discharges of \SI{1.5}{cumecs} and \SI{0.8}{cumecs} respectively. It is tidally dominated, although the contributions of both these rivers were included.

The Estuary is modelled using RMA-2, a two-dimensional depth averaged numerical model that calculates water surface elevation and horizontal velocity components at each timestep \parencite{king1990program}.
The governing equation that is solved is a two-dimensional shallow water equation (2D-SWE). The 2D-SWE is a formulation of the Reynolds averaged Navier-Stokes equation, obtained by integrating it over the vertical coordinate. The 2D-SWE also assumes a hydrostatic pressure distribution, and that the horizontal extent of the model is much larger than the vertical one \parencite{cea2006numerical}.
It uses an implicit finite element numerical integration scheme on an unstructured mesh, and because of the integration scheme used it is numerically stable. Turbulence is modelled using an eddy viscosity approximation. Friction is modelled using the Manning bed friction coefficient.

\begin{figure}[hbp]
    \centering
    \includegraphics[width=9cm]{geogg101_screenshots/practical1/fowey_estuary_map_qgis}
    \caption{Location map of the Fowey Estuary study area, southwest England, UK. }
    \label{fig:fowey_estuary_map}
\end{figure}


\section{Methods}

\subsection{Model}

The model used was two-dimensional depth averaged tidal model named RMA-2 \parencite{king1990program}. It was built within the Surface-water Modelling System (SMS) modelling program, which was used to create the mesh, integrate the bathymetry, apply the river and tidal boundary conditions and control the model runs.

\subsection{Model setup}

\subsubsection{Mesh}
A mesh of the Estuary was created by digitizing the coastline and banks of the rivers. This was resampled at \SI{30}{m} to give an element size of the same order. Beyond the mouth of the Estuary a larger element size of \SI{50}{m} was used. The meshed area, as well as a close-up of a typical section and the transition at the mouth of the Estuary, can be seen in Figure \ref{fig:combined_meshes}. In total, XX \SI{30}{m} elements and YY \SI{50}{m} elements were used. Before running the model, a quality check of the mesh was performed by SMS. There were XX `size changed' warnings, but apart from these the mesh met SMS' quality criteria.


\begin{figure}[hbp]
    \centering
    \includegraphics[width=10cm]{geogg101_screenshots/practical1/combined_meshes}
    \caption{Mesh generated for the Fowey Estuary, with a close-up of two sections. The main mesh shows the boundary conditions for both of the rivers in \SI{}{m.s^{-1}} in red, as well as the initial water level at the sea boundary shown in \SI{}{m} in blue.}
    \label{fig:combined_meshes}
\end{figure}

\subsubsection{Bathymetry}
The bathymetry used was \ldots. The bathymetry was interpolated onto the mesh using \dots.

\subsubsection{Boundary conditions}
At the influx of the Fowey and Lerryn Rivers, mean discharges perpendicular to the boundaries of \SI{1.5}{cumecs} and \SI{0.8}{cumecs} respectively were specified. At the sea boundary, water level was prescribed using an observed time series of tidal levels for the Estuary.

\subsubsection{Run control}
The model was run for \SI{74}{hours} with a \SI{15}{minute} timestep. It was run from a coldstart, with the initial water level at the sea boundary being set to \SI{-0.72}{m}. 

\subsection{Calibration}

Throughout the calibration, turbulence was modelled using an eddy viscosity of \SI{9000}{ N.s.m^{-2} }.
Following \textcite{piedra2007residual, sousa2007hydrodynamic}, the  Manning bed friction coefficient was varied to achieve the best fit between the model and the observations. The potential range for the Manning coefficient, $n$, was 0.01 to 0.04. The Root Mean Squared Error (RMSE) and Nash-Sutcliffe Efficiency (NSE, see \textcite{nash1970river}) were calculated for each value of $n$, with RMSE closer to 0 and NSE closer to one representing better fit.

\begin{align*}
    RMSE &= \frac{1}{N} \sqrt{ \sum_{t = 1}^N ( O_t - M_t) ^ 2 }\\
    NSE  &= 1 - \frac{\sum_{t = 1}^N ( O_t - M_t) ^ 2 }{\sum_{t = 1}^N ( O_t - O_{mean}) ^ 2 }
\end{align*}

Where $O_t$ is the observed value at timestep $t$, $M_t$ is the measured value, and $N$ is the number of timesteps. The results can be seen if Figure \ref{fig:calibration_stats}. From these four graphs it is clear that the value of $n$ that gives best fit with the observed data is $n = 0.025$, and this value was used for the subsequent models. The two values of $NSE = 0.956, 0.997$ both represent `excellent' fit with data according to the guidelines provided by \textcite{henriksen2008assessment} because they are greater than $0.85$. 


\begin{figure}[hbp]
    \centering
    \includegraphics[width=15cm]{geogg101_screenshots/practical1/calibration_stats}
    \caption{RMSE and NSE for velocity magnitude and water level when comparing the simulated and observed values at observation point XXX. The blue dot represents the best fit for each of the four graphs, which is given by a bed friction coefficient of $n = 0.025$.}
    \label{fig:calibration_stats}
\end{figure}

\subsection{Dredging scenario}

To evaluate the effect of dredging on the Estuary, the bathymetry is altered to simulate dredging by \SI{8}{m}. The proposed dredged channel is \SI{9}{m} deeper than the current sea floor near the harbour with a mean difference of \SI{8}{m}, and is \SI{200}{m} wide. This alteration to the bathymetry can be seen in Figure \ref{fig:delta_depth}.

\begin{figure}[hbp]
    \centering
    \includegraphics[height=6cm]{geogg101_screenshots/practical1/delta_water_depth_rotated}
    \caption{Difference in bathymetries between undredged and dredged scenarios.}
    \label{fig:delta_depth}
\end{figure}

\newpage
\section{Results and analysis}

\subsection{Existing hydrodynamic regime}

Water velocity is maximum for this Estuary during a rising tide, and reaches a magnitude of \SI{8.2}{m.s^{-1}}. The location of the maximum velocity is where the Estuary narrows to \SI{120}{m}. It is not the shallowest or narrowest part of the Estuary, but it is on a bending section and it is nearer the mouth than the narrowest part. The location of the fastest part of the Estuary for a falling tide is in a similar place, as can be seen from Figures \ref{fig:global_velocities} and \ref{fig:local_velocities}.

\begin{figure}[!h]
    \centering
    \includegraphics[width=15cm]{geogg101_screenshots/practical1/global_vec_velocity_mag}
    \caption{Maximum velocities for rising (left) and falling (right) tides. Scale on left is in \SI{}{m.s^{-1}}.}
    \label{fig:global_velocities}
\end{figure}

\begin{figure}[!h]
    \centering
    \includegraphics[width=15cm]{geogg101_screenshots/practical1/local_vec_velocity_mag}
    \caption{Local maximum velocities for rising (left) and falling (right) tides showing the fastest point of the river. Scale on left is in \SI{}{m.s^{-1}}.}
    \label{fig:local_velocities}
\end{figure}

\subsection{Effects of dredging}

After dredging, the main change to the hydrodynamic regime is a reduced velocity in the Inlet and Berth areas. This can be seen in Figure \ref{fig:dredging_delta_vel_mag}, where the velocity magnitude at both the Inlet and Berth are substantially reduced during both rising and falling tides. The reduction is biggest at the Berth. Both Mid-estuary and Upstream locations show essentially no change in velocity magnitude under the dredging scenario. For all four locations, the surface water height is almost identical in both scenarios, and hence is not shown.

\begin{figure}[!h]
    \centering
    \includegraphics[width=\linewidth]{geogg101_screenshots/practical1/dredging_delta_vel_mag}
    \caption{Difference between original and dredged velocity magnitude, with surface water level shown for reference.}
    \label{fig:dredging_delta_vel_mag}
\end{figure}

There is an additional effect of dredging: the appearance of a localised area of increased velocity just upstream of the dredging. This is shown in TODO: %\ref{fig:dredging_hotspot}. 

\section{Discussion}

TODO: finish off. Talks about how bathymetric uncertainty can be used to cal model: \textcite{cea2012bathymetric}

Wetting and drying.

Constant discharge.

Eddy viscosity.

\section{Conclusions}

% From handout: bullet points will do here.

\begin{itemize}
    \item The Fowey Estuary is a macrotidal Estuary that is suitable for being modelled by a two-dimensional shallow water model.
    \item It best modelled by RMA-2 when using a Manning bed friction coefficient of 0.025 and an eddy viscosity of \SI{9000}{ N.s.m^{-2} }.
    \item The rising tide produces faster water speeds than the falling tide. During the rising tide, peak water speed is \SI{8.2}{m.s^{-1}}.
    \item Dredging the Estuary from the Berth out to sea would reduce the maximum speed by \SI{0.15}{m.s^{-1}} at the Berth, and \SI{0.10}{m.s^{-1}} at the Inlet.
    \item Dredging the Estuary would have little to no effect further upstream of the dredged channel.
\end{itemize}

\printbibliography[filter=practical1]

\newpage
\setcounter{section}{0}
\setcounter{figure}{0}

% Practical 2:
\lhead{BTTM8 - SURFACE WATER MODELLING - EFFECTS OF CLIMATE CHANGE ON A CATCHMENT}
\part*{Modelling the effects of land use change and climate change on a catchment area}

\section{Introduction}

The purpose of this study is to calibrate and validate a model of the Karup Catchment in Denmark. This model will then be used to simulate how the catchment will bahave in 2050 under three future emissions scenarios of climate change: Low, Medium and High.  

The Karup Catchment is \SI{440}{km^2}. It is situated in western Denmark and drains into the Skive Fjord. Geologically, it consists of highly permeable sand and gravel as well as occasionally lenses of clay. 
The majority of land use is for agriculture (67\%), with forest (18\%), heath (10\%), wetland (4\%) and urban (1\%) making up the remainder \parencite{refsgaard1997parameterisation}.

\begin{figure}[!h]
    \centering
    \includegraphics[height=8cm]{geogg101_screenshots/practical2/karup_catchment_with_boreholes}
    \caption{Location of Karup Catchment in Denmark, with the main rivers and the four boreholes used for observation shown. Adapted from \textcite{blasone2008uncertainty}.}
    \label{fig:karup_catchment}
\end{figure}

MIKE SHE is a catchment model that, in one form or another, has been in use for the last 30 years \parencite{refsgaard2010systeme}. It is a deterministic, distributed and physically based modelling system for modelling water flow through catchments \parencite{refsgaard1995mike}. It can model precipitation, surface runoff, evapotsranspiration, and saturated zone and unsaturated zone subsurface flow. To model river flow it can be coupled to MIKE 11, which models one-dimensional water flow through channels by solving the Saint Venant equations \parencite{havno1995mike}.  
MIKE SHE uses a finite difference approach to modelling catchments, and can be used with various spatially distributed characteristics such as elevation, land use and soil type. Additionally, the precipitation and evapotsranspiration can be specified for each grid cell in the model \parencite{thompson2012modelling}.

\newpage
\section{Methodology}

\subsection{Model setup}

The model used grid cells of size \SI{500}{m} x \SI{500}{m}.

\subsubsection{Topography}

The topography was imported from a Digital Elevation Model (DEM) file, and was interpolated onto the model grid using bilinear interpolation. The topography can be seen in Figure \ref{fig:topography}.

\begin{figure}[!h]
    \centering
    \includegraphics[height=10cm]{geogg101_screenshots/practical2/topography}
    \caption{Topography for the region shown after it has been interpolated onto the model grid.}
    \label{fig:topography}
\end{figure}

\subsubsection{Land cover}

Four land types were defined for the model: grass crops, forest, wetlands and shrubs. Forest, wetlands and shrubs had predefined Leaf Area Indices (LAIs) and Root Depths (RDs) shown in Table \ref{table:lai_rd}. Grass crops' LAIs and RDs varied over the course of a year as shown by Figure \ref{fig:grass_variation}. The spatial distribution of the four land types can be seen in Figure \ref{fig:vegetation}.

\setlength\extrarowheight{3pt}
\begin{table}[!h]
    \centering
    \begin{tabular}{l c c}
	Land type   & LAI & RD (\SI{}{mm}) \\
	    \hline
	    Forest & \num{6} & \num{800} \\
	    Wetlands & \num{4} & \num{500} \\
	    Shrubs & \num{2} & \num{440} \\
    \end{tabular}
    \caption{LAI and RD for three of the land types.}
    \label{table:lai_rd}
\end{table}


\begin{figure}[!h]
    \centering
    \includegraphics[width=\textwidth]{geogg101_screenshots/practical2/grass_variation}
    \caption{Grass crops' LAI and RD over the full course of the simulation.}
    \label{fig:grass_variation}
\end{figure}

\begin{figure}[!h]
    \centering
    \includegraphics[width=9cm]{geogg101_screenshots/practical2/vegetation}
    \caption{Spatial distribution of the four land types. The indices in the key are: 1 - Grass crops, 2 - Forest, 3 - Shrubs and 4 - Wetlands. }
    \label{fig:vegetation}
\end{figure}


\subsubsection{Saturated zone}

In MIKE SHE the saturated zone (SZ) comprises of layers and lenses. One layer, a sand aquifer, was defined, as well as one set of clay lenses. Both of these were defined through the use of a DEM, similar to how the topography was set up. The horizontal extent of the clay lenses can be seen in Figure \ref{fig:clay_horizontal_extent}. Various hydrological properties were assigned for the aquifer and the clay lenses, such as horizontal and vertical hydraulic conductivities (see Section \ref{sec:calibration} for more details on the values used for the hydraulic conductivities).

\begin{figure}[!h]
    \centering
    \includegraphics[width=9cm]{geogg101_screenshots/practical2/clay_horizontal_extent}
    \caption{Horizontal extent of the clay lenses, which corresponds to index 1 in the key.}
    \label{fig:clay_horizontal_extent}
\end{figure}

\subsubsection{Unsaturated zone}

The Unsaturated Zone (UZ) is made up of three soil types: fine sand, coarse sand and a mixture of fine/course sand. For each soil type, a series of layers at different depths was defined, going down to \SI{37}{m}. The levels and number of layers for each depth can be seen in Table \ref{table:uz_depths}. The way these layers are defined, with the highest density of layers found nearest the surface, means that the subsurface layer receives the most computational effort during the model run.

\setlength\extrarowheight{3pt}
\begin{table}[!h]
    \centering
    \begin{tabular}{c c c}
	Depth (\SI{}{m})  & Layer height (\SI{}{m}) & Number of layers \\
	    \hline
	    1 & 0.1 & 10 \\
	    3 & 0.2 & 10 \\
	    6 & 0.3 & 10 \\
	    10 & 0.4 & 10 \\
	    37 & 0.5 & 54 \\
    \end{tabular}
    \caption{Layers of the UZ, with the number of layers and their height shown.}
    \label{table:uz_depths}
\end{table}



\subsubsection{Rivers}

Rivers are added into MIKE SHE through a coupling to MIKE 11, a dedicated one-dimensional river model. The rivers were added to the model by digitising the path of the river from a map of the area. Five branches were defined: Karup River, Haderup River, Bording Creek, Feldborg Creek and Hauge Creek. Each branch was set up with a chainage and a cross section. In this study, the coupling to MIKE 11 was one way, from MIKE SHE to MIKE 11. That is, all drainage from MIKE SHE flows into MIKE 11 rivers, but if a river were to overflow, it would not flood the surrounding area in MIKE SHE.

\subsubsection{Precipitation and evapotsranspiration}

Precipitation was defined for the nine areas shown in Figure \ref{fig:precip_areas}. Each area contained a measuring station, which provided precipitation values in the form of a time series for those grid cells nearest to it. Evapotransipiration was defined with a single time series, which provided values for this variable over the entire catchment.

\begin{figure}[!h]
    \centering
    \includegraphics[width=9cm]{geogg101_screenshots/practical2/precip_areas}
    \caption{Precipitation areas based on the measuring station in the catchment.}
    \label{fig:precip_areas}
\end{figure}



\subsection{Calibration and validation}
\label{sec:calibration}

MIKE SHE is typically calibrated by modifying the parameters listed in Table \ref{table:calibration_parameters} \parencite{refsgaard1997parameterisation, thompson2004simulation}. Following \textcite{klemevs1986operational}, the split-sample method was used. The calibration period was from 01/01/1971 to 31/12/1975, and the validation period was from 01/01/1976 to 31/12/1980. One year, 1970, was used for model spin-up time. The model was run with a timestep of (TODO: Not sure find out!).

The calibration procedure used was to modify one parameter at a time, looking for the best fit with observations at the four boreholes and two river discharge stations by examining the Root Mean Squared Error (RMSE), Nash-Sutcliffe Efficiency (NSE, see \textcite{nash1970river}) and Pearson's correlation coefficient (R). Once a value for e.g. ``Horizontal hydraulic conductivity of sand aquifer'' had been chosen, the same procedure would be used for the next parameter in the table: ``Vertical hydraulic conductivity of sand aquifer''. The chosen values for each of the four parameters can be seen in \ref{table:calibration_parameters}, and the corresponding summary statistics are tabulated in Table \ref{table:summary_stats}.

\setlength\extrarowheight{3pt}
\begin{table}[!h]
    \centering
    \begin{tabular*}{\textwidth}{@{\extracolsep{\fill} } l c c c}
	& Minimum & Maximum & Chosen \\
	Parameter   & (\SI{}{m.s^{-1}}) & (\SI{}{m.s^{-1}}) & (\SI{}{m.s^{-1}}) \\
	    \hline
	    Horizontal hydraulic conductivity of sand aquifer  & \num{1e-4} & \num{9e-4} & \num{3.9e-4} \\
	    Vertical hydraulic conductivity of sand aquifer    & \num{1e-5} & \num{9e-5} & \num{8e-5}\\
	    Horizontal hydraulic conductivity of clay lenses   & \num{1e-6} & \num{9e-6} & \num{1e-6}\\
	    Vertical hydraulic conductivity of clay lenses     & \num{1e-7} & \num{9e-7} & \num{2e-7}\\
    \end{tabular*}
    \caption{Calibration parameters, the range of appropriate values, and the chosen value.}
    \label{table:calibration_parameters}
\end{table}

\newpage
\section{Results}

\subsection{Modelled river regime}

According to the guidelines provided by \textcite{henriksen2008assessment}, an NSE greater than $0.85$ should be considered as having an `excellent' fit with the observed data. For the validation period and the two discharge stations in Table \ref{table:summary_stats}, their values for NSE are $0.97$ and $0.92$, so the fit with the observed data is `excellent'

\begin{table}[!h]
    \centering
    \resizebox{\textwidth}{!}{\begin{minipage}{\textwidth}
    \begin{tabular*}{\textwidth}{@{\extracolsep{\fill} } l c c c c c c}
	& \multicolumn{3}{c}{Calibration} & \multicolumn{3}{c}{Validation} \\
		    & RMSE & NSE  & R    & RMSE & NSE  & R    \\
	    \hline
	Borehole 5  & 0.19 & 0.63 & 0.82 & 0.23 & 0.52 & 0.77 \\
	Borehole 35 & 0.25 & 0.78 & 0.89 & 0.19 & 0.87 & 0.95 \\
	Borehole 37 & 0.23 & 0.55 & 0.81 & 0.13 & 0.81 & 0.91 \\
	Borehole 65 & 0.26 & 0.69 & 0.84 & 0.18 & 0.90 & 0.95 \\
	    \hline
	Karup       & 0.16 & 0.95 & 0.98 & 0.17 & 0.97 & 0.99 \\
	Hagebro     & 0.06 & 0.92 & 0.97 & 0.07 & 0.92 & 0.97 \\

    \end{tabular*}
    \caption{Summary statistics comparing the observed and simulated values for the four boreholes and two discharge stations. The calibration period was 01/01/1971 - 31/12/1975 and the validation period was 01/01/1976 - 31/12/1980.}
    \label{table:summary_stats}
\end{minipage} }
\end{table}

As can be seen in Figure \ref{fig:ctrl_river_regime}, the river simulated river regime also matches the observed regime quite closely. The largest discrepancy is during August for the Hagebro discharge station, when the simulated results overestimate the monthly discharge by 8\%. In general, the model tends to overestimate the discharge in the summer months when the discharge is at its lowest. Likewise, it underestimates the discharge in the winter months when the discharge is at its highest.

\begin{figure}[!h]
    \centering
    \includegraphics[height=9cm]{geogg101_screenshots/practical2/ctrl_river_regime}
    \caption{Simulated and observed river regimes calculated over the combined calibration and validation periods.}
    \label{fig:ctrl_river_regime}
\end{figure}


\newpage

When looking at the individual graphs for the four boreholes and two discharge stations (Figures \ref{fig:borehole_cal} and \ref{fig:river_cal}), the model appears to match the observations well, both for the calibration and the validation period. (In Figure \ref{fig:river_cal} the simulated line is obscured by the observed data.)

\begin{figure}[!h]
    \centering
    \includegraphics[width=14.5cm]{geogg101_screenshots/practical2/borehole_cal}
    \caption{Simulated (blue line) plotted against observed (black crosses) for the four boreholes.}
    \label{fig:borehole_cal}
\end{figure}

\begin{figure}[!h]
    \centering
    \includegraphics[width=14.5cm]{geogg101_screenshots/practical2/river_cal}
    \caption{Simulated (blue line) plotted against observed (black crosses) for the discharge stations.}
    \label{fig:river_cal}
\end{figure}


\newpage
Scatter graphs of the boreholes and discharge station again show that the model performs reasonably well at reproducing the observed data, with relatively few outliers visible. It backs up the findings from Figure \ref{fig:ctrl_river_regime} that the model overestimates the periods of low flow (summer) and underestimates those of high (winter), which can be seen by the gradient of the lines of best fit which are less than one.

\begin{figure}[!h]
    \centering
    \includegraphics[width=12cm]{geogg101_screenshots/practical2/borehole_scatter}
    \caption{Scatter graphs of observed (horizontal axis) vs simulated (vertical axis) for each of the four boreholes and the calibration/validation periods. The blue line shows the line of best fit and the corresponding R value is shown. All values are in \SI{}{metres}.}
    \label{fig:borehole_scatter}
\end{figure}

\begin{figure}[!h]
    \centering
    \includegraphics[width=12cm]{geogg101_screenshots/practical2/river_scatter}
    \caption{Scatter graphs of observed (horizontal axis) vs simulated (vertical axis) for both rivers and the calibration/validation periods. The blue line shows the line of best fit and the corresponding R value is shown. All values are in \SI{}{cumecs}.}
    \label{fig:river_scatter}
\end{figure}

\newpage

\subsection{Effects of climate change}

To simulate climate change in 2050, three different emissions scenarios were used. These were Low, Medium and High, and for each one of these new precipitation and evapotsranspiration time series were produced (one for each station in the case of precipitation). The calibrated model was then re-run with these new input data. Under these three different emissions scenarios, the river regime deviated from its original regime, as can be seen in Figure \ref{fig:effects_of_cc}. Each of the regimes is plotted against the simulated calibration/validation regime (CVR), not the observed data. The Medium and High scenarios both show significant deviation from the CVR, with the effect being largest in the winter when the discharge is greatest. The Low scenario does show deviation from the CVR, although not nearly as markedly.

\begin{figure}[!h]
    \centering
    \includegraphics[width=\textwidth]{geogg101_screenshots/practical2/effects_of_cc}
    \caption{River regimes showing the simulated effects of climate change under the Low, Medium and High emissions scenarios for both discharge monitoring stations. For reference, the simulated calibration/validation river regime is shown (black dashes). }
    \label{fig:effects_of_cc}
\end{figure}

\section{Conclusions}

The MIKE SHE/MIKE 11 model performed well at reproducing the observations from the validation period after having been calibrated.
It had a slight tendency to overestimate the discharge in summer when the discharge was at its lowest, and underestimate it in winter when it was at its highest.
According the guidelines put forward by \textcite{henriksen2008assessment}, it performs excellently.
Under a scenario of Low emissions up to 2050, the model predicts that the average monthly discharge at both of the measuring stations will be broadly unchanged.
Under Medium and High emissions scenarios up to 2050, the model predicts that the average monthly discharge at both of the measuring stations will be higher by 10\% over the course of the year, with the biggest change in winter.


\printbibliography[filter=practical2]

\end{document}
