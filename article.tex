\documentclass{article}

% Some useful packages.
\usepackage{amsmath}
\usepackage{siunitx}
\usepackage{graphicx}

% Reduces margins substantially.
\usepackage{geometry}
\newgeometry{margin=2.5cm}

% Allows headers and footers.
\usepackage{fancyhdr}
\pagestyle{fancy}
% Get rid of annoying line under header.
\renewcommand{\headrulewidth}{0pt}

\lhead{BTTM8 - SURFACE WATER MODELLING}
\chead{}
\rhead{}
%\lfoot{}
%\cfoot{}
%\rfoot{}

% Configure References sections. 2 filters are for the 2 sections. (1/practical)
\usepackage[backend=bibtex,style=authoryear,sorting=nyt,dashed=false]{biblatex}
% Changes e.g. \parencite{jones2003} from (Jones 2003) to (Jones, 2003)
\renewcommand*{\nameyeardelim}{\addcomma\space}
\defbibfilter{practical1}{keyword=practical1}
\defbibfilter{practical2}{keyword=practical2}
\bibliography{references/references}

\begin{document}

% Third person throughout. Should mainly be in present tense.

% Practical 1:
\part*{Modelling the effects of dredging on a tidal estuary}

\section{Introduction}

The purpose of this study is to implement a surface water model of the Fowey Estuary. This model is then used to characterise the existing hydrodynamic regime of the Estuary, analyse the sensitivity of the model to changes in the bed friction and evaluate the changes to the regime under a dredging scenario. % drop sensitivity?

The Fowey Estuary is a macrotidal estuary with a mean spring tidal range of \SI{4.8}{\m} and a tidal length of \SI{12}{km} \parencite{uncles2002dependence}. It is located in Cornwall on the south coast of England (see Figure \ref{fig:fowey_estuary_map}). It is fed by two rivers, the Fowey and the Lerryn, with mean discharges of \SI{1.5}{cumecs} and \SI{0.8}{cumecs} respectively. It is tidally dominated, although the contributions of both these rivers was included.

The Estuary is modelled using RMA-2, a two-dimensional depth averaged numerical model that calculates water surface elevation and horizontal velocity components at each timestep \parencite{king1990program}.
The governing equation that is solved is a two-dimensional shallow water equation (2D-SWE). The 2D-SWE is a formulation of the Reynolds averaged Navier-Stokes equation, obtained by integrating it over the vertical coordinate. The 2D-SWE also assumes a hydrostatic pressure distribution, and that the horizontal extent of the model is much larger than the vertical one \parencite{cea2006numerical}.
It uses an implicit finite element numerical integration scheme on an unstructured mesh, and because of the integration scheme used it is numerically stable. Turbulence is modelled using an eddy viscosity approximation. Friction is modelled using the Manning bed friction coefficient.

\begin{figure}[hbp]
    \centering
    \includegraphics[width=9cm]{geogg101_screenshots/practical1/fowey_estuary_map_qgis}
    \caption{Location map of the Fowey Estuary study area, southwest England, UK. }
    \label{fig:fowey_estuary_map}
\end{figure}


\section{Methods}

\subsection{Model}

The model used was two-dimensional depth averaged tidal model named RMA-2 \parencite{king1990program}. It was built within the Surface-water Modelling System (SMS) modelling program, which was used to create the mesh, integrate the bathymetry, apply the river and tidal boundary conditions and control the model runs.

\subsection{Model setup}

\subsubsection{Mesh}
A mesh of the Estuary was created by digitizing the coastline and banks of the rivers. This was resampled at \SI{30}{m} to give an element size of the same order. Beyond the mouth of the Estuary a larger element size of \SI{50}{m} was used. The meshed area, as well as a close-up of a typical section and the transition at the mouth of the Estuary, can be seen in Figure \ref{fig:combined_meshes}. In total, XX \SI{30}{m} elments and YY \SI{50}{m} elements were used. Before running the model, a quality check of the mesh was performed by SMS. There were XX `size changed' warnings, but apart from these the mesh met SMS' quality criteria.


\begin{figure}[hbp]
    \centering
    \includegraphics[width=10cm]{geogg101_screenshots/practical1/combined_meshes}
    \caption{Mesh generated for the Fowey Estuary, with a close-up of two sections. The main mesh shows the boundary conditions for both of the rivers in \SI{}{m/s} in red, as well as the initial water level at the sea boundary shown in \SI{}{m} in blue.}
    \label{fig:combined_meshes}
\end{figure}

\subsubsection{Bathymetry}
The bathymetry used was \ldots. The bathymetry was interpolated onto the mesh using \dots.

\subsubsection{Boundary conditions}
At the influx of the Fowey and Lerryn Rivers, mean discharges perpendicular to the boundaries of \SI{1.5}{cumecs} and \SI{0.8}{cumecs} respectively were specified. At the sea boundary, water level was prescribed using an observed time series of tidal levels for the Estuary.

\subsubsection{Run control}
The model was run for \SI{74}{hours} with a \SI{15}{minute} timestep. It was run from a coldstart, with the initial water level at the sea boundary being set to \SI{-0.72}{m}. 

\subsection{Calibration}

Throughout the calibration, turbulence was modelled using an eddy viscosity of \SI{9000}{ N.s.m^{-2} }.
Following \textcite{piedra2007residual, sousa2007hydrodynamic}, the  Manning bed friction coefficient was varied to achieve the best fit between the model and the observations. The potential range for the Manning coefficient, $n$, was 0.01 to 0.04. The Nash-Sutcliffe Efficiency (NSE, see \textcite{nash1970river}) and Root Mean Squared Error (RMSE) were calculated for each value of $n$. 

\begin{align*}
    RMSE &= \frac{1}{N} \sqrt{ \sum_{t = 1}^N ( O_t - M_t) ^ 2 }\\
    NSE  &= 1 - \frac{\sum_{t = 1}^N ( O_t - M_t) ^ 2 }{\sum_{t = 1}^N ( O_t - O_{mean}) ^ 2 }
\end{align*}

Where $O_t$ is the observed value at timestep $t$, $M_t$ is the measured value, and $N$ is the number of timesteps. The results can be seen if Figure \ref{fig:calibration_stats}. From these four graphs it is clear that the value of $n$ that gives best fit with the observed data is $n = 0.03$, and this value was used for the subsequent models. The two values of $NSE = 0.956, 0.997$ both represent `excellent' fit with data according to the guidelines provided by \textcite{henriksen2008assessment} because they are greater than $0.85$. 


\begin{figure}[hbp]
    \centering
    \includegraphics[width=15cm]{geogg101_screenshots/practical1/calibration_stats}
    \caption{RMSE and NSE for velocity magnitude and water level when comparing the simulated and observed values at observation point XXX. The blue dot represents the best fit for each of the four graphs, which is given by a bed friction coefficient of $n = 0.03$.}
    \label{fig:calibration_stats}
\end{figure}

\subsection{Dredging scenario}

To evaluate the effect of dredging on the Estuary, the bathymetry is altered to simulate dredging by \SI{8}{m}. The proposed dredged channel is approximately \SI{8}{m} deeper than the current sea floor near the harbour with a mean difference of \SI{5}{m}, and is \SI{200}{m} wide. This alteration can be seen in Figure \ref{fig:delta_depth}.

\begin{figure}[hbp]
    \centering
    \includegraphics[height=6cm]{geogg101_screenshots/practical1/delta_water_depth_rotated}
    \caption{Difference in bathymetries between undredged and dredged scenarios.}
    \label{fig:delta_depth}
\end{figure}

\newpage
\section{Results and analysis}

\subsection{Existing hydrodynamic regime}

Water velocity is maximum for this Estuary during a rising tide, and reaches a magnitude of \SI{8.2}{m/s}. The location of the maximum velocity is where the Estuary narrows to \SI{120}{m}. It is not the shallowest or narrowest part of the Estuary, but it is on a bending section and it is nearer the mouth than the narrowest part. The location of the fastest part of the Estuary for a falling tide is in the same place, as can be seen from Figures \ref{fig:global_velocities} and \ref{fig:local_velocities}.

\begin{figure}[!h]
    \centering
    \includegraphics[width=15cm]{geogg101_screenshots/practical1/global_vec_velocity_mag}
    \caption{Maximum velocities for rising (left) and falling (right) tides. Scale on left is in \SI{}{m/s}.}
    \label{fig:global_velocities}
\end{figure}

\begin{figure}[!h]
    \centering
    \includegraphics[width=15cm]{geogg101_screenshots/practical1/local_vec_velocity_mag}
    \caption{Local maximum velocities for rising (left) and falling (right) tides showing the fastest point of the river. Scale on left is in \SI{}{m/s}.}
    \label{fig:local_velocities}
\end{figure}

\subsection{Effects of dredging}

After dredging, the main change to the hydrodynamic regime is a reduced velocity in the Inlet and Berth areas. This can be seen in Figure \ref{fig:dredging_delta_vel_mag}, where the velocity magnitude at both the Inlet and Berth are substanitally reduced diring both rising and falling tides. The reduction is biggest at the Berth. Both Mid-estuary and Upstream locations show essentially no change in velocity magnitude under the dredging scenario. For all four locations, the surface water height is almost identical in both scenarios.

\begin{figure}[!h]
    \centering
    \includegraphics[width=\linewidth]{geogg101_screenshots/practical1/dredging_delta_vel_mag}
    \caption{Difference between original and dredged velocity magnitude, with surface water level shown for reference.}
    \label{fig:dredging_delta_vel_mag}
\end{figure}

There is an additional effect of dredging, the appearance of a localised area of increased velocity just upstream of the dredging. This is shown in %\ref{fig:dredging_hotspot}. 

\newpage
\section{Discussion}

talks about how bathymatric uncertainty can be used to cal model: \textcite{cea2012bathymetric}

\section{Conclusions}

% From handout: bullet points will do here.

\begin{itemize}
    \item The Fowey Estuary is a macrotidal Estuary that is suitable for being modelled by a two-dimensional shallow water model.
    \item It best modelled by RMA-2 when using a Manning bed friction coefficient of 0.03 and an eddy viscosity of \SI{9000}{ N.s.m^{-2} }.
    \item The rising tide produces faster water speeds than the falling tide. During the rising tide, peak water speed is about \SI{8.2}{m/s}.
    \item Dredging the Estuary from the Berth out to sea would reduce the maximum speed by about \SI{0.15}{m/s} at the Berth, and \SI{0.10}{m/s} at the Inlet.
    \item Dredging the Estuary would have little to no effect further upstream of the dredged channel.
\end{itemize}

\printbibliography[filter=practical1]

\newpage
\setcounter{section}{0}

% Practical 2:

\part*{Modelling the effects of deforestation on a catchment area}

\section{Introduction}
The Karup catchment is \SI{440}{km^2} \parencite{refsgaard1997parameterisation}.
MIKE SHE is a deterministic, distributed and physically based modelling system for modelling water flow through catchments \parencite{refshaard1995mike}. It can be coupled to MIKE 11, which models one-dimensional water flow through channels \parencite{havno1995mike}. Wetlands can be effectively simulated by MIKE SHE \parencite{thompson2004simulation}. 

\section{Methodology}
Methods

\section{Results}
Results to go here\ldots

\section{Conclusions}
Conclusions to go here\ldots

\printbibliography[filter=practical2]

\end{document}
